\documentclass[10pt,a4paper]{report}
\usepackage[utf8]{inputenc}
\usepackage{amsmath}
\usepackage{amsfonts}
\usepackage[french]{babel}
\usepackage{amssymb}

\usepackage[cm]{fullpage}

\usepackage{listings}
\usepackage{color}
\usepackage{verbatim}
\usepackage{framed}
\usepackage{ulem}
\usepackage{pigpen}

\usepackage{graphicx}
\newcommand{\ens}[1]{\lbrace #1 \rbrace}
\newcommand{\abs}[1]{\vert #1 \vert}

\newcommand{\union}{\cup}
\newcommand{\intersection}{\cap}
\newcommand{\mand}{\wedge}
\newcommand{\mor}{\vee}

\newcommand{\bigoh}{\mathcal{O}}

\newcommand{\dessin}[1]{\begin{center}\includegraphics[scale=0.6]{images/#1.png}\end{center}}
\newcommand{\dessinS}[2]{\begin{center}\includegraphics[scale=#2]{images/#1.png}\end{center}}

\newtheorem{theorem}{Théorème}[section]
\newtheorem{lemma}[theorem]{Lemma}
\newtheorem{proposition}[theorem]{Proposition}
\newtheorem{corollary}[theorem]{Corollary}

\newcommand{\qed}{\nobreak \ifvmode \relax \else
      \ifdim\lastskip<1.5em \hskip-\lastskip
      \hskip1.5em plus0em minus0.5em \fi \nobreak
      \vrule height0.75em width0.5em depth0.25em\fi}


\newenvironment{proof}[1][Preuve]{\begin{trivlist}
\item[\hskip \labelsep {\bfseries #1}]}{\qed\end{trivlist}}
\newenvironment{definition}[1][Définition]{\begin{trivlist}
\item[\hskip \labelsep {\bfseries #1}]}{\end{trivlist}}
\newenvironment{example}[1][Exemple]{\begin{trivlist}
\item[\hskip \labelsep {\bfseries #1}]}{\end{trivlist}}
\newenvironment{remark}[1][Remark]{\begin{trivlist}
\item[\hskip \labelsep {\bfseries #1}]}{\end{trivlist}}


%% Raccourcis, histoire de ne pas devenir fou avec les notations
\newcommand{\ey}[1]{E_y\{#1\}}
\newcommand{\els}[1]{E_{LS}\{#1\}}
\newcommand{\yh}{\hat{y}}				% y hat
\newcommand{\xh}{\hat{x}}				% x hat
\newcommand{\vary}[1]{var_y\{#1\}}
\newcommand{\varyx}[1]{var_{y\vert \underline{x}}\{#1\}}
\newcommand{\varls}[1]{var_{LS}\{#1\}}

\newcommand{\pyo}{P(\overline{y})}		% P(y barre)
\newcommand{\yo}{\overline{y}}			% y overline
\newcommand{\ao}{\overline{a}}			% a overline
\newcommand{\xo}{\overline{x}}			% x overline

\newcommand{\ab}{\textbf{a}} % a bold
\newcommand{\cb}{\textbf{c}} % c bold
\newcommand{\wb}{\textbf{w}} % w bold
\newcommand{\yb}{\textbf{y}} % w bold

\newcommand{\exy}{E_{\underline{x}, y}}
\newcommand{\eyx}{E_{y \vert \underline{x}y}}
\newcommand{\ex}{E_{\underline{x}}}
\newcommand{\xu}{\underline{x}}			% x underline
\newcommand{\uu}{\underline{u}}			% u underline

\newcommand{\LS}{\text{LS}}
\newcommand{\TS}{\text{TS}}
\newcommand{\VS}{\text{VS}}

\newcommand{\remarque}[1]{\textcolor{red}{#1}} % Note sur la synthèse

\title{Synthèse Apprentissage inductif appliqué}
\author{Jean-Philippe Collette}

\makeindex
\begin{document}	
	\maketitle
	\tableofcontents
	\chapter{Introduction}

L'apprentissage consiste à

\begin{itemize}
	\item améliorer les performances d'un ordinateur dans certaines tâches, avec de l'expérience ;
	\item extraire un modèle d'un système en se basant sur les observations de ce systèmes dans certaines situations ;
	\item créer un modèle, c'est-à-dire une relation entre les variables utilisées pour décrire le système.
\end{itemize}

Les deux buts principaux de l'apprentissage sont la prédiction et la meilleure compréhension d'un système.

L'apprentissage est utilise quand il n'y a pas d'expertise humaine, quand les humains ne sont pas capables d'expliquer leur expertise, quand les solutions changent au cours du temps ou quand les solutions nécessitent d'être adaptées à des cas particuliers.

\dessinS{1}{.4}

L'exploration de données se déroulent en plusieurs étapes :

\begin{enumerate}
	\item Génération de données ;
	\item Préprocessing : normalisation des valeurs, traitement des valeurs manquantes, sélections d'une composante, etc ;
	\item Apprentissage : développement d'une hypothèse, choix d'un algorithme d'apprentissage, etc ;
	\item Validation d'hypothèse : validation croisée, déploiement du modèle, etc.
\end{enumerate}

	\section{Glossaire}
	
	\dessinS{2}{.4}

	\section{Protocoles d'apprentissage automatique}
	
	% TODO Apprentissage supervisé, non-supervisé, apprentissage par renforcement, apprentissage en mode batch vs apprentissage on-line, apprentissage semi-supervisé et transductif.  Donner un exemple de problème pratique et de méthode pour chaque type de protocole.
	
	
		\subsection{Apprentissage supervisé}
		
		L'apprentissage supervisé consiste, à partir de la base de donnée (learning sample, échantillon de test), à trouver une fonction $f$ qui prend entrée les variables du problème et qui approxime au mieux la sortie :

		$$\hat{Y} = f(X_1, X_2, X_3, X_4)$$

		Plus formellement, l'apprentissage consiste, à partir d'un échantillon d'apprentissage $\ens{(x_i, y_i) \vert i = 1, \dots , N}$, avec $x_i \in \mathcal{X}$ et $y_i \in \mathcal{Y}$, à trouver une fonction $f : \mathcal{X} \rightarrow \mathcal{Y}$ qui minimise la fonction de probabilité de perte $l : \mathcal{Y} \times \mathcal{Y} \rightarrow \mathbb{R}$ sur la distribution jointe des paires d'entrées sorties : $E_{x, y} \ens{l(f(x), y)}$.

		Cette fonction de perte $l$ prend en entrée deux sorties $Y$ et retourne 1 si elles sont équivalentes, 0 sinon.

		Lorsque la sortie est une valeur symbolique, on parle de classification. Si la sortie est une valeur numérique, on parle de régression.

		Un modèle sera déterministe s'il est parfait, c'est-à-dire s'il a une règle de classification qui ne commet par d'erreur.
		
		\subsection{Algorithme d'apprentissage}
		
		Un algorithme d'apprentissage est défini par

		\begin{itemize}
			\item une famille de modèles candidats (un espace d'hypothèses $H$),
			\item une mesure de la qualité d'un modèle, et
			\item une stratégie d'optimisation.
		\end{itemize}

		L'algorithme va ainsi, à partir de l'échantillon d'apprentissage, retourner la fonction $h$ de $H$ de meilleur qualité.
		
			\subsubsection{Exemples}
			
			% TODO
			
			\subsubsection{Modes batch et on-line}
		
			En mode batch, les échantillons sont fournis et traités ensembles pour construire un modèle. Il n'est pas nécessaire de stocker le modèle mais bien ces exemples. C'est une approche classique pour l'exploitation de données (data mining).
			
		En mode online, les échantillons sont fournis et traités un à un pour mettre à jour un modèle. Ici, il faut stocker le modèle (et non les exemples). C'est une approche fortement utilisées pour les systèmes adaptatifs.
		
			On peut cependant passer facilement d'un mode à l'autre :
		
			\begin{itemize}
				\item les échantillons peuvent être fournis ensembles, mais exploités un à un
				\item les échantillons peuvent être fournis un à un, mais sont stockés et exploités ensemble.
			\end{itemize}
		
		\subsection{Apprentissage non-supervisé}
		
		L'apprentissage non-supervisé consiste à trouver des régularités dans les données, sans indications sur des entrées et des sorties. On cherche ainsi quels sont les groupes de variables ou d'objets intéressants, et s'il y a des dépendances entre les variables.
		
			\subsubsection{Exemples}
			
			% TODO
		
		\subsection{Apprentissage semi-supervisé}
		
		Il faut distinguer deux types de données : celles qui sont labellisées des données non-labellisées. Une donnée non labellisée est une paire entrée-sortie sans valeur de sortir. Elles sont plus faciles à obtenir que des données labellisées car
		
		\begin{itemize}
			\item les labels peuvent nécessiter une intervention/expertise humaine, qui peut être chère, lente et non fiable
			\item les labels peuvent nécessiter des équipements spéciaux.
		\end{itemize}
		
		Des données non labellisées se retrouvent dans le domaine biomédical, dans l'analyse du langage, dans le parsage du langage naturel, dans la catégorisation d'image ou dans la mesure de réseau.
				
		Le but de l'apprentissage non-supervisé est d'exploiter tout type de données, labellisées ou non, et de construire de meilleurs modèles qu'en utilisant seulement un seul de ces types.
		
		On utilise pour cela de l'auto-apprentissage (on labellise les données non labellisées avec un modèle entraîné sur les données labellisées) ou des algorithmes semi-supervisés (par exemple le semi-supervised SVM : on énumère tous les labels possibles pour les données non labelisées, on apprend une SVM pour chaque labeling, et on prend celui qui donne la plus grande marge).
		
		\subsection{Apprentissage transductif}
		
		Ce type d'apprentissage est similaire à l'apprentissage supervisé, mais on a accès aux données de test dès le début, ce qu'on exploite plus. On ne veut pas simplement un modèle, on veut aussi calculer des prédictions pour des données non labellisées.
		
		Pour cela, on utilise de l'apprentissage semi-supervisé en utilisant les données à tester comme données non labellisées  pour obtenir un modèle, et on l'utilise pour faire des prédictions sur les données de test. Il existe également des algorithmes spécifiques qui évitent purement et simplement la construction d'un modèle.
		
		\subsection{Apprentissage par renforcement}
		
		Il s'agit d'un apprentissage basé sur des interactions :
		
		\dessinS{110}{.3}
		
		Le but est de choisir une séquence d'actions (une politique) qui maximise $r_0 + \gamma r_1 + \gamma^2 r_2 + \dots$, avec $0 \leq \gamma < 1$.
		
		\dessinS{109}{.3} % TODO Mettre R2D2 à la place
		
		Un système est généralement modélisé par
		
		\begin{itemize}
			\item des probabilités de transition à un état $P(s_{t + 1} \vert s_t, a_t)$
			\item des probabilités de récompense $P(r_{t + 1} \vert s_t, a_t)$ 		(processus de décision de Markov).
		\end{itemize}
		
		Si le modèle des dynamiques et e la récompense est connu, on va calculer la politique optimale à adopter par programmation dynamique.
		
		Si le modèle n'est pas connu, deux approches sont possibles :
		
		\begin{itemize}
			\item on base sur un modèle : on apprend d'abord un modèle des dynamiques et on en dérive une politique optimale
			\item on n'utilise pas de modèle : on apprend directement la politique en se basant sur les observations des trajectoires du système.
		\end{itemize}
		
			\subsubsection{Renforcement vs. supervisé}
			
			\begin{center}
			\begin{tabular}{|c|c|c|}
			\hline 
			  & Supervisé & Renforcement \\ 
			\hline 
			Mode batch & \pbox{7.5cm}{Apprentissage d'un mapping d'une entrée vers une sortie à partir d'observations de paires entrée-sortie} & \pbox{7.5cm}{Apprentissage d'un mapping d'un état vers une action à partir de tripets (état, action, récompense) observés} \\ 
			\hline 
			Mode online & \pbox{7.5cm}{(Actif learning) combinaison d'apprentissage supervisé et de sélection (online) d'instances pour labeller} & \pbox{7.5cm}{Combinaison d'apprentissage de politique avec un contrôle du système et avec la génération de trajectoires d'entraînement} \\ 
			\hline 
			\end{tabular} 
			\end{center}
			
			L'apprentissage par renforcement pourrait se réduire à de l'apprentissage supervisé si l'action optimale était connue pour chaque état. De plus, l'apprentissage supervisé est utilisé par l'apprentissage par renforcement pour modéliser les dynamiques du système et/ou les fonctions de récompense.
			
			\subsubsection{Exemples}
			
			% TODO
		
		\subsection{Apprentissage actif}
		
		Pour des données non labellisées, on veut trouver (de manière adaptive) dse exemples à labellisées afin d'apprendre un modèle précis. L'espoir est de réduire le nombre d'instances labellisées en utilisant l'apprentissage supervisé de type batch.
		
		Généralement, on a une configuration de type online : on choisit les $k$ meilleurs exemples non labellisés, on détermine leurs labels, on met à jour le modèle et on itère.
		
		Les algorithmes diffèrent dans la façon dont les exemples non labellisés sont choisis.
		

	
	\part{Apprentissage supervisé}
	\chapter{Introduction}
L'apprentissage supervisé consiste, à partir de la base de donnée (learning sample, échantillon de test), à trouver une fonction $f$ qui prend entrée les variables du problème et qui approxime au mieux la sortie :

$$\hat{Y} = f(X_1, X_2, X_3, X_4)$$

Plus formellement, l'apprentissage consiste, à partir d'un échantillon d'apprentissage $\ens{(x_i, y_i) \vert i = 1, \dots , N}$, avec $x_i \in \mathcal{X}$ et $y_i \in \mathcal{Y}$, à trouver une fonction $f : \mathcal{X} \rightarrow \mathcal{Y}$ qui minimise la fonction de probabilité de perte $l : \mathcal{Y} \times \mathcal{Y} \rightarrow \mathbb{R}$ sur la distribution jointe des paires d'entrées sorties : $E_{x, y} \ens{l(f(x), y)}$.

Cette fonction de perte $l$ prend en entrée deux sorties $Y$ et retourne 1 si elles sont équivalentes, 0 sinon.

Lorsque la sortie est une valeur symbolique, on parle de classification. Si la sortie est une valeur numérique, on parle de régression.

Un modèle sera déterministe s'il est parfait, c'est-à-dire s'il a une règle de classification qui ne commet par d'erreur.

\section{Sélection de modèle}

Un algorithme d'apprentissage est défini par

\begin{itemize}
	\item une famille de modèles candidats (un espace d'hypothèses $H$),
	\item une mesure de la qualité d'un modèle, et
	\item une stratégie d'optimisation.
\end{itemize}

L'algorithme va ainsi, à partir de l'échantillon d'apprentissage, retourner la fonction $h$ de $H$ de meilleur qualité.


\section{Comparaison des méthodes}
	
\dessin{23}

A noter que l'importance relative des critères dépend de l'application, et que ce ne sont que des tendances générales.

	\chapter{Arbres de décision}

Il s'agit d'un algorithme d'apprentissage qui peut gérer les problèmes de classification (binaire ou avec plusieurs valeurs) et avec des attributs qui peuvent être discrets ou continus.

Un arbre de décision est un arbre où

\begin{itemize}
	\item chaque noeud intérieur teste un attribut,
	\item chaque branche correspond à la valeur d'un attribut, et
	\item chaque feuille est labellisée par une classe.
\end{itemize}

\dessin{16}

	\section{Création d'un arbre de décision}
	
	On a l'algorithme suivant, pour une procédure \textit{learn\_dt($LS$)}, où $LS$ est l'échantillon d'apprentissage.
	
	\begin{itemize}
		\item[$\bullet$] si tous les objets de LS ont la même classe, créer une feuille avec cette classe comme label,
		\item[$\bullet$] sinon,
		\begin{itemize}
			\item trouver le meilleur attribut $A$ pour une séparation,
			\item créer un noeud de test pour cet attribut, et
			\item pour chaque valeur $a$ de $A$,
			\begin{itemize}
				\item[$\circ$] construire $LS_a = \ens{o \in LS \vert A(o) = a}$, et
				\item[$\circ$] utiliser \textit{learn\_dt($LS_a$)}, pour créer un sous-arbre à partir de $LS_a$.
			\end{itemize}
		\end{itemize}
	\end{itemize}
	
	Pour trouver le meilleur attribut, il faut définir un score afin d'évaluer les séparations possibles. Ce score devra favoriser la séparation en classe, afin de réduire la profondeur de l'arbre.
	
	\dessin{17}
	
	Une mesure assez commune est celle-ci :
	
	$$I(LS, A) = H(LS) - \frac{\vert LS_{\text{left}}\vert}{\vert LS\vert}H(LS_{\text{left}}) - \frac{\vert LS_{\text{right}}\vert}{\vert LS \vert}H(LS_{\text{right}})$$
	
	\dessin{18}
	
	Pour éviter l'overfitting, on a trois façons :
	
	\begin{itemize}
		\item pre-pruning/pré-élagage : arrêter d'étendre l'arbre plus tôt, avant qu'il n'atteigne le point où il classe parfaitement l'échantillon d'apprentissage ;
		\item post-pruning/post-élagage : permettre à l'arbre de surapprendre et de l'élager ensuite ;
		\item les méthodes Ensemble.
	\end{itemize}
	
	\dessin{19}
	
	\section{Variables numériques}
	
	Deux solutions :
	\begin{itemize}
		\item pré-discrétiser, assigner des valeurs symboliques à des ranges (par exemple "froid" si la température est inférieure à 70$\,^{\circ}$F, "normal" si entre 70 et 75$\,^{\circ}$F, "chaud" si plus de 75$\,^{\circ}$F) ;
		\item discrétiser durant l'opération de construction de l'arbre.
		
		\dessin{20}
	\end{itemize}
	
	\subsubsection{Arbre de régression}
	
	Un arbre de régression est un arbre de décision où les labels des noeuds sont numériques.
	
	\section{Interprétabilité et sélection d'attribut}
	
	Un arbre de décision est très interprétable, il peut être converti facilement en un ensemble de règles "si \dots alors".
	
	Si certains attributs ne sont pas nécessaire pour la classification, il n'apparaitront pas dans l'arbre (élagé/pruned). C'est important si la mesure de certaines variables est coûteuse.
	
	Les arbres de décision sont souvent utilisés comme pre-processing pour d'autres algorithmes d'apprentissage, qui souffrent de variables inutiles.
	
	Certaines variables ont une importance, elles ne contribuent pas toutes de manière égale. Grâce aux arbres, on peut évaluer leur importance.
	
	$\longrightarrow$ Comment ?
	
	\section{Avantages et inconvénients}
	
	\begin{itemize}
		\item[+] très rapide et scalabe, on peut traiter d'énorme quantité d'entrées et d'objets ;
		\item[+] donne une bonne interprétabilité et quantifie l'importance des variables ;
		\item[-] grande variance ;
		\item[-] souvent pas aussi précis que d'autres méthodes.
	\end{itemize}
	
	\chapter{$k$-NN - méthode des $k$ème plus proches voisins}
		
Cette méthode consiste à prédire la sortie en se basant sur les plus proches voisins de l'entrée.
		
\dessin{11}
		
Pour ce faire, on trouve les $k$ plus proches voisins, en utilisant la distance euclidienne. La sortie sera,
		
\begin{itemize}
	\item dans le cadre d'une classification, la classe la plus fréquente,
	\item dans le cadre d'une régression, la valeur moyenne.
\end{itemize}
		
\dessin{12}
		
\begin{itemize}
	\item[+] très simple ;
	\item[+] peut être adapté pour tout type de données, en changeant la mesure de la distance ;
	\item[-] choisir une bonne mesure de la distance est un problème compliqué ;
	\item[-] cet algorithme est très sensible à la présence de bruit ;
	\item[-] lent.
\end{itemize}	
	\chapter{Méthodes linéaires}

	\section{Introduction}
	Le but est de trouver un modèle qui est une combinaison linéaire des entrées.

	\begin{itemize}
		\item Pour une régression, $y = w_0 + w_1 x_1 + w_2 x_2 + \dots + w_n x_n$
		\item Pour une classification, $y = c_1$ si $w_0 + w_1 x_1 + w_2 x_2 + \dots + w_n x_n > 0$, $c_2$ sinon.
	\end{itemize}

	\dessin{13}

	Plusieurs méthodes existent pour trouver les coefficients $w_i$ : 

	\begin{itemize}
		\item Régression : least-square regression, ridge regression, partial least square, support vector regression, LASSO, \dots
		\item Classification : linear discriminant analysis, PLS-discriminant analysis, support vector machines, \dots \\
	\end{itemize} 

	Avantages et inconvénients :
	
	\begin{itemize}
		\item[+] simple ;
		\item[+] il existe des variantes rapides et scalables ;
		\item[+] cette méthode offre des modèles interprétatifs, à travers des poids variables (magnitude et signe) ;
		\item[-] parfois pas aussi précis que d'autres méthodes (non linéaires).
	\end{itemize}

	\section{Ridge regression}
	
	Une régression linéaire essaie d'approximer la sortie avec
	
	$$\yh(o) = w_0 + \sum_{i = 1}^n w_i a_i(o)$$
	
	Si on pose $a_0(o) = 1 \: \forall o$, et si on dénote
	
	\begin{itemize}
		\item $\ab'(o) = (a_0(o), a_1(o), \dots , a_n(o))^T$, et
		\item $\wb' = (w_0, w_1, \dots , w_n)^T$,
	\end{itemize}
	
	On a 
	
	$$\yh(o) = \wb'^T\ab'(o)$$
	
	Le carré de l'erreur (Square Error) sur un échantillon $i$ peut s'écrire :
	
	$$SE(o_i, \wb') = (y(o_i) - \yh(o_i))^2 = (y(o_i) - \wb'^T\ab'(o_i))^2$$
	
	On a aussi le carré de l'erreur sur tous les objets de $LS$ (Total Square Error), avec $A' = (\ab'_1, \dots , \ab'_N)$ et $\yb = (y(o_1), y(o_2), \dots , y(o_N))$ :
	
	$$TSE(LS, \wb') = \sum_{i = 1}^N (y(o_i) - \wb'^T\ab'(o_i))^2 = (\yb - A'^T\wb')^T(\yb - A'^T\wb')$$
	
	avec
	
	$$A' = \begin{array}{c}\left. \underbrace{\begin{pmatrix}
	1 & 1 & \dots & 1 \\ 
	a_1(o_1) & a_2(o_1) & \dots & a_N(o_1) \\ 
	\vdots & \dots & \dots & \vdots \\ 
	a_1(o_n) & a_2(o_n) & \dots & a_N(o_n)
	\end{pmatrix}}_{N}\right\} n + 1\end{array} $$
	
		
	On cherche $\wb'$ qui minimise
		
	$$\TSE(\LS, \wb') = (\yb - A'^T\wb')^T(\yb - A'^T\wb')$$
		
	Le gradient est
		
	$$\Delta_{w'} TSE(\LS, \wb') = -2 A' (\yb - A'^T\wb')$$
		
	Si on résout $\Delta_{w'} TSE(\LS, \wb') = 0$, on obtient
	
	\begin{eqnarray*}
	& 0 = & -2A'\yb + 2A'A'^T\wb' \\
	\Leftrightarrow & 2A'\yb = & 2A'A'^T\wb' \\
	\Leftrightarrow & \wb' = & (A'A'^T)^{-1}A'\yb
	\end{eqnarray*}
		
		
	Il existe cependant plusieurs solutions optimales, ce n'est pas un problème à solution unique. Afin d'avoir une solution unique, on va introduire un terme $\lambda > 0$ de régulation : l'erreur régulée vaut
	
	$$\TSE_R(\LS, \wb') = (\yb - A'^T\wb)^T(\yb - A'^T\wb') + \textcolor{red}{\lambda \wb'^T\wb'}$$
	
	Le gradient devient
	
	$$\Delta_{w'} TSE_R(\LS, \lambda, \wb') = -2 A' (\yb - A'^T\wb') + \textcolor{red}{2 \lambda \wb'}$$
	
	La solution optimale devient alors
	
	$$\wb^*(\lambda) = (A' A'^T + \textcolor{red}{\lambda I})^{-1} A \yb$$
	
	Cette solution est unique $\forall \lambda > 0$. Augmenter $\lambda$ réduit la variance, car on tient moins compte de $A'$ dans $TSE_R$. En revanche, on est moins optimal sur l'échantillon, il y a augmentation de l'erreur quadratique moyenne.
		
	\dessin{39}
		
	Augmenter $\lambda$ est bénéfique sur un échantillon de test indépendant.
	
	En terme de temps de calcul :
	
	\begin{itemize}
		\item créer la matrice de covariance est de l'ordre de $N n^2$ opérations
		\item résoudre le système pour trouver $\wb^*$ est de l'ordre de $n^3$ opérations
		\item[$\rightarrow$] $\bigoh(n^3)$
	\end{itemize}
	
	
	C'est la méthode de régulation qui donne son nom à la \textit{ridge regression}
	
	
\section{Perceptron}
	
L'intuition derrière le perceptron est qu'un cerveau humain peut apprendre, et qu'on pourrait s'en inspirer pour développer un algorithme. Ainsi, un perceptron est la modélisation d'un neurone, qui va ensuite former des couches pour créer des réseaux de neurones.
	
\dessinS{111}{.3}
	
La sortie est une somme pondérée des entrées. Le seul paramètre à adapter au problème est $\wb'$/$\wb$, qui doit minimiser
	
$$\sum_i (y_i - wx_i)^2$$

en utilisant une descente de gradient : étant donné un exemple d'entrainement $(x, y)$,

$$\delta \leftarrow y - w^Tx$$
$$w_j \leftarrow w_j + \eta \delta x_j \: \forall j$$

Il s'agit d'un algorithme de type \textit{online}, c'est-à-dire qu'il traite les exemples un à un, alors qu'un algorithme de type \textit{batch} traite tous les exemples en une seule fois.

La complexité est régulée par le taux d'apprentissage $\eta$ et le nombre d'itérations.

	\subsection{Algorithme du perceptron}
	
	Pour de la classification binaire, on pose que $c(o) = \pm 1$. On définit $\eta_i$ le taux d'apprentissage.
	
	\begin{itemize}
		\item on commence avec un vecteur de poids initial arbitraire, par exemple $\wb_0' = \textbf{0}$.
		\item on considère chaque élément de $\LS$ dans une séquence cyclique ou aléatoire
		
		\item soit l'objet $o_i$ l'objet à l'étape $i$, $c(o_i)$ est sa classe et $\ab(o_i)$ son vecteur d'attributs
		\item si l'objet $o_i$ est mal classé, on ajuste le vecteur de poids avec la règle suivante :
		
		$$\wb_{i + 1}' = \wb_i' + \eta_i (c(o_i) - g(\ab(o_i))) \ab'(o_i)$$
	\end{itemize}
	
	% Slide 9/21 : vue géométrique, que je n'arrive pas à comprendre
	
	\subsection{Soft threshold units}
	
	La fonction d'entrée/sortie $g(\ab)$ est donnée par
	
	$$g(\ab(o)) \triangleq f(w_0 + \wb^T\ab(o)) = f(\wb'^T \ab'(o))$$
	
	$f(.)$ est une fonction d'activation qui est supposée dérivable. On a généralement pour cette fonction
	
	\begin{itemize}
		\item une sigmoïde :
		
		$$sigmoid(x) = \frac{1}{1 + \exp(-x)}$$
		
		\dessin{118}
		
		\item une tangente hyperbolique :
		
		$$tanh(x) = \frac{\exp(x) - \exp(-x)}{\exp(x) + \exp(-x)}$$
		
		\dessin{117}
		
		\item la fonction $sgn$
		
		\dessin{119}
		
	\end{itemize}
	
	\subsection{Descente de gradient}
	
	Slides 11 et 12/21
	
	\subsection{Propriétés}
	
	Si $\LS$ est linéairement séparable, l'algorithme convergera en un nombre fini d'étape. Sinon, il convergera avec un nombre infini d'étape si $\eta_i \rightarrow 0$.
	
	Résultats théoriques de l'algorithme de descente de gradient : 13/21
	
	\subsection{Couche de perceptrons}
	
	On considère des perceptrons mis côte à côte. Ils ont tous en entrée tous les attributs des objets et fonctionnent de manière indépendante. Par exemple, on pourrait avoir le réseau suivant qui permet d'effectuer une classification parmi quatre classes ; chaque neurone code un bit de la classe.
	
	\dessinS{112}{.45}
	
	Cela ne fonctionne toujours que dans le cas de problèmes linéairement séparables, sinon il faut considérer plusieurs couches afin d'obtenir un modèle non linéaire

\section{Réseaux de neurones}

Les réseaux de neurones ont le pouvoir de représentation universel : on peut tout représenter (à un $\sigma$ près) pour autant qu'on a suffisamment de neurones et de couches.

Un réseau est composé d'au moins une couche d'entrée et d'une couche de sortie, les couches situées entre les deux sont appelées les couches cachées. Chaque neurone est relié à tous les neurones de la couche précédente.

	\subsection{Classification}
	
	On considère généralement trois couches :
	
	\begin{enumerate}
		\item la première peut définir un ensemble d'hyper/demi-plans
		\item la seconde peut définir des intersections d'hyper/demi-plans
		\item la troisième peut définir des unions d'intersections
	\end{enumerate}
	
	\dessinS{113}{.65}
	
	\subsection{Régression}
	
	On considère deux couches :
	
	\begin{itemize}
		\item la première définit $K$ paramètres d'offset $\beta_i$ et de scale $\alpha_i$ : les réponses sont $f(\alpha_i x + \beta_i)$ avec $i = 1, \dots , K$
		\item la seconde couche est une linéarisation :
		
		$$\yh(x) = b_0 + \sum_{i = 1}^K b_i f(\alpha_i x + \beta_i)$$
	\end{itemize}
	
	\dessinS{114}{.5}
	
	Plus on a de neurones et plus on peut approximer des fonctions continues, avec un $K$ suffisamment grand. La phase d'apprentissage permet d'ajuster les paramètres des fonctions.
	
	\subsection{Apprentissage d'un réseau multi-couche}
	
	On utilise la propagation arrière des dérivations : on commence l'apprentissage à partir de la dernière couche pour revenir à la première. L'idée est de calculer les dérivations de la fonction d'erreur (au sens des moindres carrés) en partant de la couche de sortie jusqu'à la couche d'entrée, ce qui a pour but de réduire la fonction d'erreur en modifiant les poids.
	
	On a l'algorithme suivant pour un réseau de trois couches : % From http://en.wikipedia.org/wiki/Backpropagation
	
	\begin{itemize}
		\item initialiser les poids du réseau (généralement aléatoirement)
		\item tant que les exemples ne sont pas correctement classés ou qu'un critère n'est pas satisfait :
		
		\begin{itemize}
			\item pour chaque exemple $e$ de l'ensemble d'apprentissage
				\begin{itemize}
					\item soit $o$ la sortie du réseau de neurones pour l'entrée $e$ (forward pass) et $t$ la "vraie" sortie de $e$ (donnée de l'ensemble d'apprentissage)
					\item calculer l'erreur $t - o$
					\item calculer les deltas pour tous les poids des synapses qui vont de la couche cachée à la couche de sortie (backward pass)
					\item calculer les deltas pour tous les poids des synapses qui vont de la couche d'entrée vers la couche cachée (backward pass continued)
					\item mettre à jour les poids du réseau
				\end{itemize}
		\end{itemize}
	\end{itemize}
	
	
\section{Méthodes de régression à base de noyaux}
	
Un noyau $K(., .)$ est une fonction qui exprime la similarité de deux objets à travers une valeur numérique. Pour tout noyau $K$ il existe une fonction $\phi$ telle que $K(o, o') = \phi(o) \times \phi(o')$.

Lorsqu'on utilise la régression au sens des moindres carrés avec des noyaux, on a un modèle de la forme

$$\yh(x) = w^T \phi(x) + b$$

où on cherche $w$ et $b$ qui minimisent la fonction d'erreur

$$\text{Err}(w, b) = \frac{1}{2} w^T w + \frac{1}{2} \sum_{k = 1}^N \underbrace{(y_k - (w^T \phi(x_k) + b))^2}_{e_k}$$

C'est un problème d'optimisation sous contraintes. En passant par le lagrangien on a

\begin{eqnarray*}
w & = & \sum_{j = 1}^N \alpha_j \phi(x_j) \\
\alpha_k & = & y e_k \\
\sum_{k = 1}^N \alpha_k & = & 0
\end{eqnarray*}

On a donc

$$e_k = y_k - (b + \sum_{j = 1}^N \alpha_j K(x_j, x_k))$$

Si on réintroduit $e_k$ dans l'équation 2, on a

$$\yh(x) = b + \sum_{i = 1}^N \alpha_i K(x, x_i)$$

avec $K$ une matrice définie par $K_{i, j} = K(x_i, x_j)$. Cela revient à écrire

$$\begin{pmatrix}
0 & 1 & \dots & 1 \\ 
1 &   &   &   \\ 
\vdots &   & K + \gamma^{-1}l &   \\ 
1 &   &   &  
\end{pmatrix} \begin{pmatrix}
b \\ 
\alpha_1 \\ 
\vdots \\ 
\alpha_N
\end{pmatrix} = \begin{pmatrix}
0 \\ 
y_1 \\ 
\vdots \\ 
y_N
\end{pmatrix} $$


L'algorithme d'apprentissage va alors utiliser les valeurs de la matrice $K$ et $y_k$ pour déterminer les $\alpha_i$ et $b$. Au final, on n'a pas besoin de connaître $\phi(x)$.

Au niveau de la complexité, on a 

\begin{itemize}
	\item $N^3$ opérations pour l'apprentissage
	\item $N$ opérations pour la prédiction
\end{itemize}

	


%%%%%%%%%%%%%%%%%%%%%%%%%%%%%%%%%%%%%%%%%%%%%%%%%%%%%%%%%%%%%
%   _____                                          _      
%  / ____|                                        | |     
% | (___   ___  _   _ _ __ ___ ___    ___ ___   __| | ___ 
%  \___ \ / _ \| | | | '__/ __/ _ \  / __/ _ \ / _` |/ _ \
%  ____) | (_) | |_| | | | (_|  __/ | (_| (_) | (_| |  __/
% |_____/ \___/ \__,_|_|  \___\___|  \___\___/ \__,_|\___|
% | |              | |                                    
% | |__   __ _  ___| | ___   _ _ __                       
% | '_ \ / _` |/ __| |/ / | | | '_ \                      
% | |_) | (_| | (__|   <| |_| | |_) |                     
% |_.__/ \__,_|\___|_|\_\\__,_| .__/                      
%                             | |                         
%                             |_| 
% 
%%%%%%%%%%%%%%%%%%%%%%%%%%%%%%%%%%%%%%%%%%%%%%%%%%%%%%%%%%%%%
		
	% Ce qui suit est un résumé des slides du cours à partir du 10/19. Vu que ce n'est pas spécialement accessible ni amusant, le lecteur peut aller en parler à Denver à la fin de ce fichier.
		
	\begin{comment}
	[slide 10/19]
		
	Considérons la matrice $(A'A'^T)$ : l'élément $i, j$ est obtenu par le produit scalaire de la $i$ème ligne et de la $j$ ème ligne de $A'$ :
		
	$$A'A'^T = N \begin{pmatrix}
		1 &  \ao_1 & \dots & \ao_n \\ 
		\ao_1 & g_{1,1} & \dots & g_{1, n} \\
		\vdots & \vdots & \ddots & \vdots \\
		\ao_n & g_{n, 1} & \dots & g_{n, n}
		\end{pmatrix} $$
		
	avec $\ao_i = \frac{1}{N} \sum_{k = 1}^N a_i(o_k)$ et $g_{i, j} = \frac{1}{N} \sum_{k = 1}^N a_i(o_k)a_j(o_k)$.
		
	Si la moyenne est nulle, $\ao_i = 0$ et les $g_{i, j}$ forment la matrice de covariance $\Sigma$
		
	\remarque{slides 12-19/19 ; ce qui suit vient de mes notes manuscrites}
		
	En conclusion, la prédiction n'est pas modifiée si on ajoute des constantes et si on effectue des combinaisons linéaires aux valeurs des attributs (tant qu'elles ne sont pas singulières).
		
	Si on retire la moyenne et l'écart-type à tous les coefficients de la matrice, on les place dans le même ordre de grandeur (abstraction des unités).
				
	Le problème est que plusieurs solutions sont possible. La solution est d'utiliser $\lambda$ [slide 16/19], qui permet d'en isoler une et de mieux conditionner le problème.
		
	Augmenter $\lambda$ réduit la variance, car on tient moins compte de $A$ dans $TSE_R$. En revanche, on est moins optimal sur l'échantillon, il y a augmentation de l'erreur quadratique moyenne.
		
	\dessin{39}
		
	Augmenter $\lambda$ est bénéfique sur un échantillon de test indépendant.
		
	\end{comment}
		
		
		
	% Ce qui suit appartiennait à une première version de la synthèse, qui commencait par résumer l'overview.
	
	\begin{comment}
	\section{Extensions non linéaires}

	Plusieurs extensions existent :

	\begin{itemize}
		\item la généralisation des méthodes linéaires :
	
		$$y = w_0 + w_1 \phi_1(x_1) + w_2 \phi_2(x_2) + \dots + w_n \phi_n(x_n)$$
	
		N'importe quelle méthode linéaire peut être appliquée, mais la régulation devient plus importante.
	
		\item Réseaux de neurones artificiels, avec une seule couche cachée : si $g$ est une fonction non linéaire (par exemple une sigmoid),
	
		$$y = g(\sum_j w_J g(\sum_i w_{i, j} x_i))$$
	
		C'est une fonction non linéaire d'une combinaison linéaire de fonction non linéaires de combinaisons linéaires d'entrées.
	
		\item Méthode à base de noyaux :
	
		$$y = \sum_i w_i \phi_i(x) \Leftrightarrow y = \sum_j \alpha_j k(x_j, x)$$
	
		où $k(x, x') = \langle \phi(x), \phi(x') \rangle$, le produit scalaire dans l'espace donné et où $j$ indexe les exemples d'entraînement du modèle.
	\end{itemize}

	\end{comment}
		
% Ce qui suit vient d'une première synthèse basée sur l'overview. Ce n'est plus pertinent de l'inclure dans cette synthèse, mais le code est conservé au cas où il faudrait effectuer quelques prélèvements.

\begin{comment}
\subsection{Modèle linéaire}

\dessin{3}

On a par exemple

$$h(X_1, X_2) = \left\{
\begin{array}{ll}
\text{malade} & \text{si } w_0 + w_1X_1 + w_2X_2 > 0 \\
\text{sain} & \text{sinon.}
\end{array}
\right.$$

La phase d'apprentissage aura pour but de trouver les meilleurs $w_0$, $w_1$ et $w_2$ ; un modèle linéaire ne possède que trois paramètres.

\subsection{Modèle quadratique}
\dessin{4}

$$h(X_1, X_2) = \left\{
\begin{array}{ll}
\text{malade} & \text{si } w_0 + w_1X_1 + w_2X_2 + \textcolor{red}{w_3 X_1^2 + W_4X_2^2} > 0 \\
\text{sain} & \text{sinon.}
\end{array}
\right.$$

La phase d'apprentissage aura pour but de trouver les meilleurs $w_0$, $w_1$, $w_2$, $w_3$ et $w_4$.

Par facilité, on peut étendre la base de données, en ajoutant des colonnes contenant $X_1^2$ et $X_2^2$.

Un modèle quadratique est au moins aussi bon que le meilleur modèle linéaire, car on peut rendre un modèle quadratique linéaire avec $w_3 = w_4 = 0$.


\subsection{Modèle d'un réseau de neurones artificiels}
\dessin{5}
$$h(X_1, X_2) = \left\{
\begin{array}{ll}
\text{malade} & \text{si une fonction complexe de } X_1, X_2 > 0 \\
\text{sain} & \text{sinon.}
\end{array}
\right.$$
La phase d'apprentissage aura pour but de trouver les paramètres numériques de cette fonction.

\end{comment}


%           _                  
%          //                                                  
%         //                                                   
%      __/(                                                    
%  _.~-a  ~-.                                                  
% {_____)    `.           _..=~~~~=._                          
%       ~-_    \      _.=~           '=.                       
%          \    `._.=~            .=.   :=._                   
%           -         __         (   \   : \)                  
%            ~.      (  }       (     |   : :                  
%              `:     \ \        \    |\   ; :                 
%                \     \ }        \   / |  ;  }                
%                 `-.__//__.==~~=._\ (_/  ;  ;                 
%                     //           | |/  ;  ;                  
%                    {{       _____|_/ ;   ;        *     ___  
%                     `      ---- _=.=`   ~ _____   ||*    ____
%                             __:='    .='     ___\\||/___     
%                         ..:~____.==''                        
	\subsection{Réseaux de neurones artificiels}
		
55 $\rightarrow$ 60
	\chapter{Modèle de Bayes}
	\chapter{Évaluation de modèles}

On aimerait estimer les performances d'un modèle ayant appris sur un ensemble de données (de taille $N$). Le but est de

\begin{itemize}
	\item sélectionner un ou plusieurs modèles (ex : déterminer la bonne complexité ou choisir entre différents algorithmes d'apprentissage).
	\item évaluer les modèles, afin d'estimer les performances sur des nouvelles données.
\end{itemize}


\section{Erreurs}
	\subsection{Erreur de resubstitution}
	
	L'erreur de re-substitution ($\LS$ error) est l'erreur obtenue en appliquant le modèle à l'échantillon d'apprentissage. Plus le modèle est complexe et plus cette erreur sera proche de 0.
	
	Cette erreur est un mauvais indicateur des performances d'un modèle car il est beaucoup trop optimiste.
	
	\subsection{Erreur de généralisation}

	L'erreur de généralisation est l'erreur obtenue sur la prédiction de nouvelles données. Si $\fh_{\LS}$ est la fonction apprise sur un échantillon $\LS$, l'erreur de généralisation est décrite par

	$$\exy\ens{L(y, \fh_{\LS}(x))}$$

	avec $L(., .)$ une fonction de perte qui mesure la différence entre les arguments.

	Cette erreur est à différencier de l'erreur de généralisation attendue (expected generalization error) sur un $\LS$ aléatoire de taille $N$, décrite par $E_{\LS} \ens{\text{Err}_{\LS}} = \els{\exy{L(y, \fh_{\LS}(x))}}$.

\section{Méthodes d'évaluation}
	\subsection{Méthode test set}

	On suppose qu'on dispose de beaucoup de données, que $N$ est grand. On divise la base de données en deux parties, une qui servira d'ensemble d'apprentissage et l'autre d'ensemble de test (par ex 70\%, 30\%). La méthode est la suivante :

	\begin{itemize}
		\item on apprend le modèle sur $\LS$
		\item on le test sur $\TS$
		\item l'estimation qui en résulte est une estimation de l'erreur d'un modèle qui aurait appris sur toute la base de données.
	\end{itemize}
	
	Avantages et inconvénients :
	
	\begin{itemize}
		\item[+] très simple à mettre en place ;
		\item[+] efficace ;
		\item[-] les données d'apprentissage sont moindres ;
		\item[-] instable lorsque la base de données est petite.
	\end{itemize}
	
	\dessin{65}
	
	\subsection{K-fold}
	
	La méthode du test-set n'est pas fiable sur une petite base de données car elle est basée sur un petit échantillon d'une base de données déjà réduite. De plus, on utilise l'estimation du modèle construit comme une estimation pour toute la base de données, or lorsque la base de données est très petite, apprendre le modèle sur toute la base ou sur une partie change fortement les résultats.
	
	On peut tracer la courbe d'apprentissage où on confronte les performances à la taille de l'échantillon d'apprentissage. On voit qu'il faut que le $\LS$ soit d'une taille minimale si on veut avoir des résultats pertinents.
	
	\dessin{63}
	
	On utilise pour cela la validation k-fold : on divise aléatoirement la base de données en $k$ sous-ensembles (typiquement $k = 10$).
	
	\dessin{64}
	
	La méthode est la suivante :
	
	\begin{itemize}
		\item pour chaque sous-ensemble
		\begin{itemize}
			\item apprendre le modèle sur les objets qui ne sont pas dans le sous-ensemble
			\item calculer les prédictions du modèle sur les points du sous-ensemble
		\end{itemize}
		\item reporter l'erreur moyenne sur ces prédictions
	\end{itemize}
	
	Lorsque $k = N$, la méthode s'appelle une validation croisée \textit{leave-one-out}.
	
	Le choix de $k$ est très important et conduit à différents avantages et inconvénients :
	
	\begin{itemize}
		\item si $k = N$ (leave-one out) :
		\begin{itemize}
			\item[+] non-biaisé : enlever un objet ne change pas trop la taille de l'échantillon d'apprentissage
			\item[-] grande variance : on dépend énormément de la base de données
			\item[-] lent : il faut entraîner $N$ modèles
		\end{itemize}
		\item si $k = 5, 10$ :
		
		\begin{itemize}
			\item[+] petite variance et rapide : on n'a que $5-10$ modèles sur peu de données
			\item[-] potentiellement biaisé (voir courbe d'apprentissage)
		\end{itemize}
	\end{itemize}
	
	
\subsection{Impact de la complexité d'un modèle avec une validation croisée}

	\dessin{7}

	Il vaut mieux utiliser une petite complexité si on ne dispose pas de beaucoup d'échantillons.

	Avec une complexité fixe, on obtient le comportement suivant.

	\dessin{8}

	Le contrôle de la complexité s'appelle la régulation ou le lissage (smoothing). Il peut être contrôlé de plusieurs façons :

	\begin{itemize}
		\item en variant la taille de l'espace d'hypothèse, autrement dit le nombre de modèles candidats, la valeur des paramètres, etc ;
		\item avec un critère de performance : on oppose les performances de l'ensemble d'apprentissage et la valeur des paramètres, autrement dit on minimise
	
		$$\text{Err}(LS) + \lambda C(\text{model})$$
	
		\item avec des algorithmes d'optimisation : le nombre d'itération, la nature du problème d'optimisation, etc.
	\end{itemize}

	Le choix d'un algorithme se fait en comparant leur taux d'erreur avec une méthode de type cross-validation sur des sous-échantillons. Ensuite, l'algorithme avec le plus petit taux est utilisé comme modèle prédicatif sur toutes les données.

	L'utilisation intensive de la méthode CV peut entraîner du sur-apprentissage. En effet, plus on compare de modèles complexes, plus on a une chance d'en trouver un qui convient pour les données. La solution pour éviter cela est de réserver un ensemble de test additionnel (ou de les générer), et de l'utiliser pour tester les performances du modèle final.
	
	\subsection{Bootstrap}
	
	Un échantillon bootstrap est un échantillon avec un remplacement : des objets n'apparaissent pas et d'autres apparaissent plusieurs fois.
	
	\dessin{66}
	
	On a alors que
	
	$$P(o_i \in \text{ bootstrap}) = 1 - (1 - \frac{1}{N})^N \approx 1 - \frac{1}{e} = 0.632$$
	
	L'idée est d'utiliser les 30\% de données qu'il reste comme $\TS$. On peut alors estimer l'erreur de bootstrap :
	
	\begin{itemize}
		\item pour $i = 1$ jusqu'à $B$ :
		
		\begin{itemize}
			\item prendre un échantillon de boostrap $B_i$ de la base de données
			\item apprendre un modèle $f_i$ sur cet échantillon
		\end{itemize}
		
		\item pour chaque objet, calculer l'erreur de tous les modèles qui ont été construis sans lui (environ 30\%)
		\item moyenner sur tous les objets
	\end{itemize}
	
	Des améliorations existent :
	
	\begin{itemize}
		\item $.632$ bootstrap : correction pour la courbe d'apprentissage
		\item $.632+$ bootstrap : correction pour le sur-apprentissage
	\end{itemize}
	
	\subsection{Erreurs de test conditionnelles et erreurs de test attendues}
	
	Pour un modèle $\hat{f}_\LS$ donné, on a l'erreur de test conditionnelle :
	
	$$\text{Err}_\LS = \exy{L(y, \hat{f}_\LS(x))}$$
	
	On a l'erreur attendue :
	
	$$\els{\text{Err}_\LS} = \els{\exy{L(y, \hat{f}_\LS(x))}}$$
	
	Seule la méthode test set estime la première erreur, la validation croisée permet quant à elle d'estimer la seconde car on fait de l'apprentissage sur des $\LS$ différents.
	
\section{Méthodes de sélection}

Le but, pour une base de données de $N$ objets, est de déterminer le meilleur modèle possible et d'estimer l'erreur des prédictions. La méthodologie dépend encore une fois de la taille de la base.

Le choix d'une mesure de l'erreur ou de la qualité dépend fortement de l'application. On peut également définir des autres critères pour évaluer un modèle.

	\subsection{Méthode test set}
	
	Si la base de données est grande, on divise aléatoirement l'ensemble d'apprentissage en trois parties : un ensemble d'apprentissage $\LS$, un ensemble de validation $\VS$ et un ensemble de test $\TS$ (par exemple 50\%, 25\% et 25\%).
	
	\dessin{67}
	
	La méthode est la suivante :
	
	\begin{itemize}
		\item apprendre les modèles à comparer sur $\LS$
		\item sélectionner le meilleur en basant les performances sur $\VS$
		\item le ré-entraîner sur $\LS + \VS$
		\item le tester sur $\TS$, afin d'avoir une estimation des performances
		\item le ré-entraîner sur $\LS + \VS + \TS$, afin d'avoir le modèle final
	\end{itemize}
	
	$\VS$ permet de sélectionner le modèle. Une fois que c'est fait, on réapprend sur $\LS$ et $\VS$ et on teste sur $\TS$. Le choix de la meilleure méthode (apprise sur $\LS$ et testée sur $\VS$) n'entraîne pas de sur-apprentissage, mais il y a tout de même un risque s'il y a trop de méthodes à tester (on en aurait une qui donne des bons résultats par coup de chance), ce qui pourrait donner une grosse erreur sur $\TS$. La solution est d'ajouter une nouvelle boucle de validation croisée ; tout choix d'échantillon peut créer un biais.
	
	\subsection{Validation croisée}
	
	On utilise deux étapes de validation croisée en k-fold.
	
	\dessin{68}
	
	CV1 est utilisé pour évaluer le modèle final, tandis que CV2 est utilisé pour la sélection du modèle.
	
	On peut également combiner la méthode test set et la validation croisée.
	
	\dessin{69}
	
	Les deux étapes sont nécessaires, car plus on compare des modèles et plus la probabilité de tomber par chance sur celui qui donne des bons résultats augmente. Les erreurs sur $\VS$ ou sur CV2 sont alors en général trop optimistes.
	
	Illustration :
	
	\dessinS{70}{.4}
	
	\subsection{Méthodes analytiques}
	
	On cherche le modèle qui minimise un critère de la forme
	
	$$\text{Err}(\LS) + G(\text{complexité})$$
	
	où $G$ est une fonction monotone croissante. Le critère est dérivé de preuves théoriques.
	
	L'avantage est qu'il n'y a pas de re-entraînement, par contre on ne peut l'utiliser que pour la sélection de modèle, et on pourrait manquer le vrai optimum dans le cas d'échantillons finis.
	
\section{Biais de sélection}
	
	En général, n'importe quel choix fait en utilisant la sortie doit être dans une boucle de validation croisée, car il peut potentiellement amener à du sur-apprentissage. En effet, un gain d'apprentissage sur un $\LS$ peut ne pas se répercuter sur des nouvelles données à prédire.
	
	\dessinS{71}{.4}
		
\section{Mesure de performance}

	\subsection{Classification binaire}
	
	Les résultats peuvent être résumés dans un tableau de contingence/matrice de confusion.
	
	\dessinS{72}{.5}
	
	On définit alors
	
	\begin{eqnarray*}
	\text{Taux d'erreur (error rate)} & = & \frac{FP + FN}{N + P} \\
	\text{Précision (accuracy)} & = & \frac{TP + TN}{N + P} = 1 - \text{ error rate}
	\end{eqnarray*}
	
	Le taux d'erreur est cependant limité : on n'a pas d'informations sur la distribution des erreurs sur les classes, et il est sensible aux changement dans la distribution des classes dans l'échantillon de test.
	
	\dessinS{73}{.45}
	
	Dans le cadre de diagnostiques médicaux, on utilise des mesures plus appropriées :
	
	\begin{eqnarray*}
	\text{Sensibilité (sensitivity/recall)} & = & \frac{TP}{P} \\
	\text{Spécificité (specificity)} & = & \frac{TN}{TN + FP} = 1 - \frac{FP}{N}
	\end{eqnarray*}
		
	La sensibilité permet de détecter le maximum de positif (plus elle est grande et plus on est sûr de détecter les cas positifs), tandis que la spécificité détecte le maximum de négatif. L'avantage de ces mesures est qu'elles ne dépendent pas de la proportion d'objets positifs ou négatifs.
	
	Ces mesures peuvent être portées sur une courbe ROC (Receiver operating characteristic). Si la sortie de l'algorithme d'apprentissage est un nombre (par exemple, la probabilité d'appartenir à une classe), on peut utiliser un seuil afin de régler la sensibilité et la spécificité.
	
	\dessinS{74}{.5}
	
	Le meilleur algorithme est le D. Si un modèle retourne toujours la classe positive, il se trouvera dans le coin supérieur droit. S'il retourne toujours la classe négative, il sera dans le coin inférieur gauche. S'il renvoie une valeur au hasard, il sera situé au centre ; la diagonale représente les choix aléatoires biaisés.
	
	Si on se trouve dans la diagonale inférieure, on peut inverser les réponses ($+ \Rightarrow -$, $- \Rightarrow +$) pour revenir dans la diagonale supérieure.
	
	\dessinS{75}{.5}
	
	On peut résumer une courbe ROC avec un nombre : l'AUC, la surface en dessous de la courbe. On peut l'interpréter comme la probabilité que deux objets choisis aléatoirement dans l'échantillon sont correctement ordonnés par le modèle, c'est-à-dire que le positif a un plus haut score que le négatif.
	
	On utilise d'autre mesures :
	
	\begin{itemize}
		\item la précision, la proportion de bonnes prédictions parmi les prédictions positives
		\item le \textit{recall}, la proportion de positifs qui sont détectés
		\item la \textit{f-mesure}
	\end{itemize}
	
	\begin{eqnarray*}
	\text{Précision} & = & \frac{TP}{TP + FP} \\
	\text{Recall} & = & \frac{TP}{TP + FN} \\
	\text{F-mesure} & = & \frac{2 * \text{précision} * \text{recall}}{\text{précision} + \text{recall}}
	\end{eqnarray*}
	
	\dessinS{76}{.5}
	
	Le meilleur algorithme aura une courbe ROC de la forme {\pigpenfont I}, tandis que la courbe de recall aura la forme {\pigpenfont C}.
	
	\subsection{Régression}
	
		A partir du moment où on a plus d'une classe, on utilise un taux d'erreur plutôt qu'une courbe ROC.
	
		\subsubsection{Erreur quadratique}
		
		$$\frac{1}{N} \sum_{i = 1}^N (y_i - \yh_i)^2$$
		
		Le carré permet de pénaliser les très mauvaises prédictions.
		
		\subsubsection{Erreur absolue moyenne}
		
		$$\frac{1}{N} \sum_{i = 1}^N \vert y_i - \yh_i \vert$$
		
		\subsubsection{Corrélation de Pearson}
		
		$$\frac{\sum_i (y_i - \frac{1}{N} \sum_j y_j)(\yh_i - \frac{1}{N} \sum_j \yh_j)}{(N - 1)s_y s_{\yh} }$$
		 		
		\subsubsection{Corrélation de rang de Spearman}
		
		$$1 - \frac{6 \sum_i d_i^2}{N(N^2 - 1)}$$
		
		avec $d_i$ la différence de rang de $y_i$ et $\yh_i$, le rang étant l'ordre obtenu après un tri des valeurs.
		
	\subsection{Mesures de performances pour l'entraînement}
	
	Les mesures de performances pour l'entraînement peuvent être différentes des mesures de performances de test. Il y a plusieurs raisons à cela :
	
	\begin{itemize}
		\item algorithmiquement, une mesure dérivable est soumise à une optimisation de gradient (par exemple le taux d'erreur et l'erreur absolue moyenne ne sont pas dérivables, l'AUC n'est pas décomposable)
		\item le sur-apprentissage : pour l'entraînement, la perte incorpore souvent un terme de pénalité pour la complexité du modèle (ce qui est inutile au moment du test). De plus, certaines mesures sont moins promptes au sur-apprentissage (par exemple la marge).
	\end{itemize}		
	\chapter{Machines à support vectoriel}

Cette méthode se base sur deux idées :

\begin{enumerate}
	\item la mise en place d'un classifieur à large marge, et
	\item le "noyautage" de l'espace d'entrée.
\end{enumerate}

\dessin{14}
Il faut trouver un classifieur linéaire. L'idée va être de trouver celui qui maximise la marge, c'est-à-dire la largeur du bord qui peut être étendue jusqu'à toucher une donnée.

\dessin{15}

C'est une méthode intuitivement sûre, avec une borne théorique sur l'erreur : $E(TS) < \mathcal{O}(\frac{1}{\text{margin}})$.

65 $\rightarrow$ 74

\section{Machines à support vectoriel linéaire}
	
	Soit un $\LS = \ens{(x_k, y_k)}^N_{k = 1}$, avec $y_k \in {-1, 1}$ et $x_k \in \mathbb{R}^n$. On cherche un classifieur de la forme
	
	$$\yh(x) = sgn(w^T x + b)$$
	
	qui classifie $\LS$ correctement, c'est-à-dire qui minimise
	
	$$\sum_{k = 1}^N1(y_k \neq \yh(x_k))$$
	
	\subsection{Hyperplan de marge maximale}
	
	Lorsque les données sont linéairement séparables dans l'espace des features, l'hyperplan séparateur n'est pas unique.
	
	\dessin{55}
	
	Une SVM va chercher à maximiser la distance de l'hyperplan au point le plus proche dans $\LS$, autrement dit
	
	$$\max_{w, b} min \ens{\Vert x - x_k \Vert : w^T x + b = 0, k = 1, \dots , N}$$
	
	On maximise la marge car, intuitivement, c'est une méthode sûr. De plus, il existe des bornes théoriques sur l'erreur de généralisation qui dépend de la marge :
	
	$$Err(TS) < \bigoh (\frac{1}{\gamma}$$
	
	où $\gamma$ est la marge. Cependant, ces marges ne sont pas souvent atteintes. En pratique, une SVM fonctionne très bien.
	
	Cet algorithme conduit à un problème d'optimisation convexe, où la solution peut être écrite uniquement en terme de produits scalaires.
	
	\subsection{Problème d'optimisation}
	
	\dessin{56}
	
	$w$ est perpendiculaire à la ligne $y(x) = w^Tx + b$ :
	
	$$y(x_a) = 0 = y(x_b) \Rightarrow w^T(x_A - x_B) = 0$$
	
	Soit $x$ tel que $y(x) = 0$. La distance de l'origine à cette ligne est
	
	$$\Vert x \Vert \cos(w, x) = \Vert x \Vert \frac{w^T x}{\Vert w \Vert \: \Vert x \Vert} = \frac{w^Tx}{\Vert w \Vert} = \frac{-b}{\Vert w \Vert}$$
	
	Tout point $x$ peut être écrit comme
	
	$$x = x_{\text{perp}} + r \frac{w}{\Vert w \Vert}$$
	
	où $\vert r \vert$ est la distance entre $x$ et la ligne. En multipliant les deux membres par $w^T$ et en ajoutant $b$, on obtient
	
	$$w^Tx + b = w^T x_{\text{perp}} + b + r \frac{w^T w}{\Vert w \Vert} = 0 + r \Vert w \Vert \Rightarrow r = \frac{y(x)}{\Vert w \Vert}$$
	
	Le problème d'optimisation peut alors être écrit comme 
	
	$$arg \max_{w, b} \ens{\frac{1}{\Vert w \Vert} \min_n [y_n . (w^T x_n + b)]}$$
	
	La solution n'est pas unique vu que l'hyperplan est inchangé si  on multiplie $w$ et $b$ par une constante $c > 0$. Pour imposer une unicité, on choisit typiquement $\vert w^T x + b \vert = 1$ pour le point $x$ qui est le plus proche de la surface (vecteur de support).
	
	Le problème est alors équivalent à maximiser $\frac{1}{\Vert w \Vert}$ (ou à minimiser $\Vert w \Vert$) avec les contraintes
	
	$$y_k(w^T x_k + b) \geq 1, \: \forall k = 1, \dots , N$$
	
	Le problème de la SVM est équivalent à
	
	$$\min_{w, b} \varepsilon(w, b) = \frac{1}{2} \Vert w \Vert^2$$
	
	sujet aux $N$ contraintes
	
	$$y_k(w^T x_k + b) \geq 1, \: \forall k = 1, \dots , N$$

	$\Vert w \Vert$ devient $\frac{1}{2} \Vert w \Vert^2$ par facilité. Il s'agit d'un problème de programmation quadratique. Il existe une solution seulement si les données sont linéairement séparables.
	
	Optimisation des contraintes : 10 $\rightarrow$ 17/45
	
	\subsection{Vecteurs de support}
	
	Le problème primaire est donc
	
	$$\mathcal{L}(w, b, \alpha) = \frac{1}{2} \Vert w \Vert - \sum_{k = 1}^N \alpha_k (y_k (w^Tx_k + b) - 1)$$
	
	En accord avec les conditions complémentaires KKT, le vecteur solution $w$ est tel que
	
	$$\alpha_k (y_k (w^T x_k + b) - 1) = 0, \: \forall k = 1, \dots , N$$
	
	$\alpha_k = 0$ si la contrainte est satisfaite comme une inégalité stricte $y_k (w^T x_k + b) > 1$, car c'est la façon de maximiser $\mathcal{L}$.
	
	$\alpha_k > 0$ si la contrainte est satisfaite comme une égalité $y_k(w^Tx_k + b) = 1$, auquel cas $x_k$ est le vecteur de support.
	
	Une fois que les valeurs optimales de $\alpha$ ont été déterminées, le modèle final peut être écrit comme
	
	$$\yh(x) = sgn(\sum_{i= 1}^N y_i \alpha_i x_i^T x + b)$$
	
	où les valeurs de $\alpha_k$ différentes de 0 (strictement positives) correspondent aux vecteurs de support.
	
	$b$ est calculé en exploitant le fait que pour tout $\alpha_k > 0$, on a nécessairement $y_k(wTx_k + b) - 1 = 0$.
	
	Borne pour le leave-one-out : 20/45
	
	\subsection{Sort margin}
	
	A cause du bruit ou de données isolées (outliers), les données peuvent ne pas être linéairement séparables dans l'espace des features. 
	
	\dessin{57}
	
	
	\chapter{Méthodes d'ensemble}

L'idée est de combiner plusieurs modèles construits avec un algorithme d'apprentissage. Cela permet d'améliorer très fort la précision. Les arbres de décision sont souvent utilisés pour des raisons d'efficacité.

	\subsubsection{Bagging}
	
	Différents échantillons d'apprentissage conduisent à différents modèles, surtout si l'algorithme surapprend les données. Vu qu'il n'y a qu'un seul modèle optimal, la variance est la source d'erreur.
	
	La solution est d'agréger plusieurs modèles pour en obtenir un qui est stable. Plus il y en a, plus les résultats seront meilleurs.
	
	\dessin{21}
	
	Un type d'agrégation est le bootstrap aggregating : chaque modèle apprend sur un échantillon où des remplacements ont été effectué, avec des lignes redondantes.
	
	\dessin{22}
	
	
	\subsubsection{Boosting}
	
	L'idée est de combiner plusieurs modèles "faibles", afin de produire un modèle plus puissant.
	
	95 $\rightarrow$ 96
	
	\subsubsection{Interprétabilité et efficacité}
	
	Lorsque les méthodes ensemble sont combinées avec les arbres de décision, elles perdent de l'interprétabilité et de l'efficacité. En revache, on les utilise toujours pour calculer l'importance des variables, en effectuant la moyenne sur tous les arbres. De plus, les méthodes ensemble peuvent être parallélisée et l'algorithme boosting utilise des petits arbres, ce qui fait que le coût en temps processeur n'est pas important.
	
	\subsection{Comparaison des méthodes}
	
	\dessin{23}

	A noter que l'importance relative des critères dépend de l'application, et que ce ne sont que des tendances générales.
	

	
	\part{Apprentissage non supervisé}
	\chapter{Apprentissage non supervisé}

Le but de l'apprentissage non supervisé est de trouver des irrégularités dans les données, sans se soucier de la relation entrée-sortie. On recherche ainsi les groupes de variables ou d'objets intéressants, et des dépendances entre les variables.

Il existe trois grandes familles de problèmes :

\begin{itemize}
	\item clustering : trouver des groupes d'échantillons ou de variables.
	\item réduction de dimensionnalité : on projette les données d'un espace à haute dimension vers un espace plus petit.
	\item estimation de densité : déterminer la distribution des données dans l'espace d'entrée.
\end{itemize}

\section{Clustering}

Le but est de grouper une collection d'objets en sous-ensembles (appelés clusters), de façon à ce que chaque objet dans un cluster soit proche des autres, tout en étant éloigné des objets des autres clusters.
	
Ces groupements peuvent être

\begin{itemize}
	\item des groupements de lignes/d'objets similaires
	\item des groupements de colonnes/variables
	\item du bi-clustering, c'est-à-dire en se basant sur les lignes et les colonnes.
\end{itemize}

\dessin{43}

Applications :

\begin{itemize}
	\item marketing : trouver des groupes de clients qui ont un comportement similaire, en se basant sur leurs caractéristiques et les achats précédents
	\item biologie : classifier de la faune et la flore selon leurs caractéristiques
	\item web : classification de documents (par exemple des articles de blog)
\end{itemize}

Deux composantes sont considérées :

\begin{itemize}
	\item la mesure de distance entre deux objets
	\item un algorithme de clustering, qui va minimiser les distances entre les objets d'un groupe et/ou maximiser les distances entre des groupes
\end{itemize}

	\subsection{Mesure de distances}
		\subsubsection{Distance Euclidienne}
		Elle mesure la différence entre des coordonnées et pénalise les grosses différences. Il s'agit de la racine carrée de la somme des carrés des différences entre les coordonnées :
		
		$$d_e(x_1, x_2) = \sqrt{(x_{10}-x_{20})^2 + (x_{11}-x_{21})^2 + \dots}$$
		
		\subsubsection{Distance de Manhattan}
		Elle mesure la différence entre des coordonnées, mais de manière robuste. Il s'agit de la somme des différences absolues de toutes les coordonnées :
		
		$$d_e(x_1, x_2) = \vert x_{10}-x_{20} \vert + \vert x_{11}-x_{21} \vert + \dots$$
		
		\subsubsection{Corrélation}
		Elle mesure une différence en tenant compte des tendances. La distance entre deux vecteurs est $1 - \rho$, où $\rho$ est la corrélation de Pearson entre les deux vecteurs :
		
		$$\rho(x_1, x_2) = \frac{cov(x_1, x_2)}{\sigma_{x_1} \sigma_{x_2}} = \frac{\sum_{i = 1}^n (x_{1, i} - \overline{x}_1)(x_{2, i} - \overline{x}_2)}{\sqrt{\sum_{i = 1}^n (x_{1, i} - \overline{x}_1)^2} \sqrt{\sum_{i = 1}^n (x_{2, i} - \overline{x}_2)^2}}$$
	
		On a que $\rho \in [-1, 1]$, donc $1 - \rho \in [0, 2]$ : 0 signifie que les données sont fortement corrélées.
	
	\subsection{Clustering hiérarchique}
	
		\subsubsection{Algorithme}
		On a l'algorithme suivant :
	
		\begin{enumerate}
			\item Chaque objet est assigné à son propre cluster
			\item Itérativement :
		
			\begin{itemize}
				\item les deux clusters les plus similaires sont joins et rassemblés en un.
				\item la matrice de distances est mise à jour avec le nouveau cluster qui en remplace deux.
			\end{itemize}
		\end{enumerate}
		
		\subsubsection{Distance entre deux clusters}
		
		On a plusieurs possibilités :
		
		\begin{itemize}
			\item Single linkage : utiliser la plus petite distance entre deux objets du cluster. Cela a tendance à créer des clusters étalés.
			
			\item Complete linkage : utiliser la plus grande distance entre deux objets du cluster. Cela a tendance à créer des grappes.
			
			\item Average distance : calculer la distance moyenne. On obtient un mix des deux autres mesures ; on a une sorte de distance entre les centres de masse.
		\end{itemize}
				
		\dessin{44}
		
		\dessin{45}
		
		\subsubsection{Dendrogramme}
		
		Cela permet de visualiser le clustering hiérarchique et de déterminer visuellement le nombre de clusters.
		
		\dessin{46}
		
		
		\subsubsection{Forces et faiblesses}
		
		\begin{itemize}
			\item[+] on n'a pas besoin de supposer un nombre particulier de cluster
			\item[+] on peut utiliser n'importe quel type de matrice de distance
			\item[+] on a parfois une interprétation facile des résultats
			
			\item[-] trouver une interprétation n'est pas toujours aisé
			\item[-] une fois qu'il a été décidé de combiner deux clusters, on ne peut pas revenir en arrière. Par exemple, le cluster rouge montre un mauvais départ, alors qu'on aurait voulu obtenir les deux clusters verts :
			
			\dessin{47}
			\item[-] pas très bien motivé théoriquement
		\end{itemize}
		
		\subsubsection{Algorithme de clustering combinatoire}
		
		Soit un nombre de clusters $K < N$ et un encodeur $C$ qui assigne la $i$ème observation au cluster $C(i)$. On va chercher la fonction $C^*$ qui minimise une fonction de perte, qui mesure si l'objectif de clustering est atteint.
		
		Par exemple, on pourrait avoir comme fonction de perte une qui se base sur l'éparpillement des objets d'un cluster (within cluster scatter) :
		
		$$W(C) = \frac{1}{2} \sum_{k = 1}^K \sum_{C(i) = k} \sum_{C(i') = k} d(x_i, x_{i'})$$
		
		Plus les points d'un cluster sont rapprochés, plus ce nombre est petit et donc plus les clusters sont meilleurs.
		
		Le nombre de possibilité est cependant trop grand pour une énumération.
		
		$$S(N, K) = \frac{1}{K!} \sum_{k = 1}^K(-1)^{K - k} \begin{pmatrix}
		K \\ 
		k
		\end{pmatrix}  k^N$$
	
	\subsection{K-means}
	
	Cet algorithme effectue un partitionnement avec un nombre $k$ fixé de clusters. On utilise la distance euclidienne entre deux objets, et on va chercher à minimiser la somme des variances intra-cluster :
	
	$$W(C) = \frac{1}{2} \sum_{k = 1}^K \sum_{C(i) = k} \sum_{C(i') = k} \Vert x_i - x_{i'} \Vert^2$$ 
	
	Le fait d'utiliser une distance euclidienne permet de réécrire la fonction et d'être plus efficace en passant d'un calcul quadratique par rapport au nombre de cluster à un calcul linéaire sur les points :
	
	$$W(C) = \sum_{k = 1}^K N_k \sum_{C(i) = k} \Vert x_i - \overline{x}_k \Vert^2$$
	
	avec $\overline{x}_k = (\overline{x}_{1k}, \dots, \overline{x}_{pk})$ le centre du cluster $k$ et $N_k$ le nombre de points dans le cluster $k$ :
	
	$$N_k = \sum_{i = 1}^N I(C(i) = k)$$
	
	Cela revient donc à un problème d'optimisation :
	
	$$min_{C, \ens{m_k}^K_1} \sum_{k = 1}^K N_k \sum_{C(i) = k} \Vert x_i - m_k Vert^2$$
	
	On a deux degrés de liberté : les clusters des points et les centres de masse $m_k$.
	
	Si $C$ est fixé, les $m_k$ sont triviaux à calculer. Si les centres de masse sont fixés, on va chercher les plus proches voisins :
	
	$$C(i) = argmax_k \Vert x_i - m_k \Vert^2$$
	
	
		\subsubsection{Algorithme}
		
		\begin{enumerate}
			\item On assigne aléatoirement chaque point à un cluster OU on répartit les centres de masse aléatoirement
			\item Itérativement :
			
			\begin{itemize}
				\item calculer les moyennes des clusters $\ens{m_1 , \dots , m_K}$
				\item pour ces moyennes, assigner chaque observation à la moyenne de cluster la plus proche :
				
				$$C(i) = argmin_{1 \leq k \leq K} \Vert x_i - m_k \Vert^2$$
			\end{itemize}
		\end{enumerate}
		
		On s'arrête lorsqu'il n'y a plus de changement ; à chaque itération, la somme va diminuer.
		
		\dessin{48}
		
		\dessin{49}
		
		\dessin{50}
		
		Chaque étape réduit l'éparpillement dans les clusters, donc la convergence est assurée, mais uniquement vers un optimum local. De plus, chacune de ces étapes est linéaire par rapport au nombre d'objets, alors qu'avec le clustering hiérarchique il fallait considérer toutes les paires d'objets.
		
		Vu le départ aléatoire de l'algorithme, on pourrait avoir plusieurs solutions différentes. La solution est de redémarrer l'algorithme plusieurs fois.
		
		L'algorithme des k-means peut être utilisé pour de la compression, en découpant une image en blocs et l'appliquant dessus.
		
		\dessin{54}
		
		\begin{center}
		\begin{tabular}{|c|c|c|c||c|c|}
		\hline
		$p_1$ & $p_2$ & $p_3$ & $p_4$ & C & $m_k$ \\ 
		\hline\hline
		0 & 255 & 128 & 255 & 1 & $m_1$ \\ 
		\hline 
		$\vdots$ & $\vdots$ & $\vdots$ & $\vdots$ & 2 & $m_2$ \\ 
		\hline 
		  &   &   &   & 1 & $\vdots$ \\ 
		\hline 
		 &   &   &   & 3 &  \\ 
		\hline 
		 &  &  &  & 4 &  \\ 
		\hline 
		\end{tabular} 
		\end{center}
		
		\subsubsection{K-medoids}
		
		Il s'agit d'une extension des k-means qui permet d'utiliser n'importe quelle mesure de distance. 
		
		\dessin{51}
		
		Les k-medoids sont par contre beaucoup plus lents : dans le calcul de l'expression, pour chaque cluster, on doit vérifier toutes les paires de point, ce qui donne un algorithme quadratique.
		
		\subsubsection{Forces et faiblesses}
		
		\begin{itemize}
			\item[+] simple et facile à comprendre
			\item[+] on peut clusteriser n'importe quel nouveau point (contrairement au clustering hiérarchique)
			\item[+] bonne motivation théorique
			\item[-] il faut fixer le nombre de clusters
			\item[-] sensible au choix initial des centres des clusters
			\item[-] sensible aux données isolées (outliers)
		\end{itemize}
	
	\subsection{Self-Organizing Maps}
	
	C'est une méthode similaire au k-means, mais avec des contraintes supplémentaires : les clusters sont donnés sous forme de matrice (à une ou deux dimensions).
	
	\dessin{52}
	
		\subsubsection{Algorithme}
		
		Itérativement :
		\begin{itemize}
			\item prendre $P$ données aléatoirement		
			\item déplacer tous les noeuds dans la direction de $P$ : plus un noeud est dans la topologie mieux c'est (et inversement)
			\item diminuer la quantité de mouvement autorisée
		\end{itemize}
		
		
	\subsection{Nombre de cluster}
	
	La question est de savoir quand s'arrêter dans le clustering hiérarchique et comment choisir $k$ pour les k-means et les SOMs. On a un phénomène d'overfitting, comme en apprentissage supervisé :
	
	\begin{itemize}
		\item on sur-apprend les données s'il y a trop de clusters. On arrive ainsi à trouvers des clusters qui n'existent pas dans les données à cause du bruit
		\item on sous-apprend les données s'il y a trop peu de clusters, on passe ainsi à côté de clusters pertinents
	\end{itemize}
	
	Il n'est cependant pas possible, contrairement à l'apprentissage supervisé, d'effectuer une validation croisée.
	
	Un choix de nombre de cluster consiste à prendre le point d'inflexion de la courbe de variance intra-cluster.
	
	\dessin{53}
	
	
	Autres possibilités :
	
	\begin{itemize}
		\item utiliser des indices internes : sélectionner $k$ qui minimise/maximise les distances intra- et extra-cluster
		\item gap statistic : utiliser une méthode qui compare un indice interne avec ce qu'on aurait obtenu avec des données aléatoire, et choisir le $k$ qui minimise cette différence
		\item stabilité : sélectionner le $k$ qui conduit à des clusters stables (calculés avec une analyse bootstrap)
	\end{itemize}
	
	\subsection{Sélection de features}
	
	La sélection de composantes peut améliorer le clustering en diminuant le bruit (et le temps de clacul).
	
	Par exemple, on va garder les 100 variables avec le plus de variance.
	
	Du clustering après une sélection de features supervisée doit être évitée, car on retrouvera toujours la classification, vu que c'est le critère utilisé pour sélectionner les variables.
	

\section{Réduction de dimensionnalité}

On va chercher à diminuer la dimensionnalité de l'espace des données. On a plusieurs possibilités :

\begin{itemize}
	\item une sélection de features : on trouve un sous-ensemble des variables originales : $X_i' = X_j$ pour certains $j$
	\item une extraction de fetures : on transforme l'espace original en un espace de plus petite dimension : $X_i' = f(X_1, \dots , X_p)$
	\item des méthodes linéaires : $f(X_1, \dots , X_p) = w_0 + w_1 X_1 + \dots + w_p X_p$
\end{itemize}

Dans tous les cas, on doit pouvoir reconstruire la base de données d'origine à partir de la version compacte. Les objectifs de la réduction de dimensionnalité sont de

\begin{itemize}
	\item réduire la dimensionnalité avant de faire passer les données dans d'autres méthodes
	\item choisir les variables les plus utiles $\Leftrightarrow$ qui apportent le plus d'information
	\item compresser les données
	\item visualiser des données multidimensionnelles, pour identifier des groupes d'objets et identifier des \textit{outliers}.
\end{itemize}

	\subsection{PCA}
	
	L'analyse en composantes principales (PCA) est une méthode linéaire qui transforme un grand nombre de variables en un petit ensemble de variables non corrélées que l'on appelle composantes principales. L'idée est de mapper les points dans de petites dimensions, tout en préservant au maximum la variances des données.
	
	On va chercher une direction (composante principale) afin, après projection des données, de maximiser l'étalement. On peut prendre une deuxième composante, afin de conserver plus d'informations, au cas où on aurait trop d'approximations. Chaque point est ainsi codé par sa distance par rapport à une origine.
	
	\dessin{78}
		
	\dessin{79}
	
	Cette méthode est très efficace quand il y a beaucoup de corrélation entre les variables (donc de la redondance).
	
	\dessin{83}
	
	\subsubsection{Approche mathématique}
	
	On a deux formulations du problème possibles :
	
	\begin{itemize}
		\item maximisation de la variance : on trouve les directions qui maximisent la variance des données projetées.
		\item minimisation de l'erreur : on minimise l'erreur de reconstruction des données projetées.
	\end{itemize}
	
	Considérons un ensemble d'observations $\ens{x_n}$, $n = 1, \dots , N$, avec $x_n$ un vecteur de dimension $D$. On veut trouver la direction unitaire $u_1$ qui maximise la variance de la projection :
	
	$$arg \max_{u_1} \frac{1}{N} \sum_{n = 1}^N \Vert u_1^T x_n - u_1^T \xo \Vert^2 = u_1^TCu_1$$
	
	avec
	
	$$\Vert u_1 \Vert = u_1^Tu_1 = 1$$
	$$C = \frac{1}{N} \sum_{i = 1}^N (x_n - \xo)(x_n - \xo)^T$$
	
	$\Vert u_1^T x_n - u_1^T \xo \Vert^2$ est la variance du vecteur considéré, car on a
	
	$$(\sum_i u_1^Tx_i - \underbrace{\frac{1}{N} \sum_i u_1^Tx_i}_{-u_1^T\sum_i \underbrace{\frac{1}{N} u_1^Tx_i}_{\xo}})^2$$
	
	La contrainte de vecteur unitaire ($\Vert u_1 \Vert = 1$) est nécessaire, sinon le vecteur aurait une variance infinie.
	
	Si on introduit le lagrangien :
	
	$$u_1^T Cu_1 + \lambda_1(1 - u_1^Tu_1)$$
	
	$$\frac{\vartheta}{\vartheta u_1} = 0 \Leftrightarrow Cu_1 - \lambda u_1 = 0 \Leftrightarrow Cu_1 = \lambda u_1$$
	
	$u_1$ est un vecteur propre de $C$. La variance est donnée par
	
	$$\rightarrow u_1^TCu_1 = \lambda u_1^Tu_1 = \lambda \Vert u_1 \Vert = \lambda_1$$
	
	On a donc que $u_1$ est un vecteur propre correspondant à la plus grande valeur propre $\lambda_1$.
	
	
	Rechercher la seconde composante est similaire, mais on ajoute une contrainte d'orthogonalité. Si on généralise à $M$ composantes, on a que la $M + 1$ème composante est obtenue en maximisant 
	
	$$u_{M + 1}^T C u_{M + 1}$$
	
	avec les contraintes
	
	$$u_{M + 1}^T u_{M + 1} = 1$$
	$$u_{M + 1} u_i = 0, \: \forall i = 1, \dots , M + 1$$
	
	Avec le multiplicateur lagrangien :
	
	$$u_{M + 1}^T Cu_{M + 1} + \lambda_{M + 1} (1 - u_{M + 1}^T u_{M + 1}) + \sum_{i = 1}^M \nu_i u_{M + 1}^T u_i$$
	
	On a l'optimum
	
	$$0 = 2 C u_{M + 1} - 2 \lambda_{M + 1} u_{M + 1}  +\sum_{i=1}^M \nu_i u_i$$
	
	Si on multiplie à gauche par $u_i^T$, on a
	
	$$0 = 2u_i^T C u_{M + 1} - 2 \lambda_{M + 1} \underbrace{u_i^T u_{M + 1}}_{0}  +\sum_{i=1}^M \nu_i u_i^T u_i$$
	
	Or, 
	$$u_i^TCu_{M + 1} = (C^Tu_i)^Tu_{M + 1} = (Cu_i)^T u_{M + 1} = (\lambda_i u_i)^T u_{M + 1} = \lambda_i u_i^T u_{M + 1} = 0$$
	
	On a donc que $\nu_i = 0$ et donc
	
	$$C u_{M + 1} = \lambda_{M + 1} u_{M + 1}$$
	
	$u_{M + 1}^T$ est le vecteur propre de la $M + 1$ème plus grande valeur propre.
	
	La $i$ème composante principale pour des objets $x_k$ est donnée par $x_{ji}' = u_i^Tx_j$. La reconstruction de l'entrée se fait par
	
	$$\xu_j = \sum_{i = 1}^M x_{ji}' u_i = \sum_{i = 1}^M(u_i^Tx_j)u_i$$
	
	PCA minimise également l'erreur de reconstruction :
	
	$$arg \max_{u_1, \dots , u_M} \frac{1}{N} \sum_{i = 1}^N \Vert x_j - \xh_j \Vert^2$$
	
	\dessin{80}
	
	Au final, on obtient un tableau de la forme
	
	\begin{center}
	\begin{tabular}{c|cccc}
	\hline 
	  & 1 & 2 & \dots & $k$ \\ 
	\hline 
	$\xu_1$ & $\uu_1^T\xu_1$ & $\uu_2^Tx_1$ & \dots & $\uu_k^T\xu_1$ \\ 
	$\xu_2$ &   &   &   &   \\ 
	\vdots &   &   &   &   \\ 
	\end{tabular} 
	\end{center}
	
	Les $u_k$ sont les vecteurs propres de la matrice de covariance, les valeurs propres sont quant à elle les variances de la projection.
	
	\subsubsection{Etude des composantes}
	
	\dessin{81}
	
	Le poids de chaque variable donne une idée de son importance dans une composante, et peut être utilisé pour de la sélection de features. Pour chaque composantes, on a une mesure du pourcentage de la variance des données initiales qu'elles contiennent.
	
	La question est de savoir combien de composantes il faut prendre. Pour cela, on trace le scree plot, les valeurs propres ($\Leftrightarrow$ variances) de chaque composante en ordre décroissant.
	
	\dessin{82}
	
	On supprime les composantes avec des valeurs propres inférieures à 1 et on prend $k$ au point d'inflexion de la courbe, à partir duquel les débris commencent à s'accumuler.
\end{document}	